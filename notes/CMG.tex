%
\documentclass{article}
\usepackage{amsmath,amssymb,bbm,amsthm}
\usepackage{fullpage}
\usepackage{thm-restate,color,xcolor,xspace}
\usepackage{hyperref,cleveref}
\usepackage{thmtools}
\usepackage{graphicx}
\usepackage{algorithm,algorithmic}



\newtheorem{theorem}{Theorem}
\newtheorem{lemma}[theorem]{Lemma}
\newtheorem{definition}[theorem]{Definition}
\newtheorem{corollary}[theorem]{Corollary}
\newtheorem{observation}[theorem]{Observation}
\newtheorem{claim}[theorem]{Claim}
\newtheorem{subclaim}{Subclaim}
\newtheorem{fact}[theorem]{Fact}
\newtheorem{assumption}[theorem]{Assumption}
\newtheorem{remark}[theorem]{Remark}

\newcommand{\ent}{\textup{entropy}}

\begin{document}


\title{CMG}
\date{\today}
\maketitle

\section{The standard CMG of KRV}

\begin{enumerate}
\item Start with $G$ as empty graph
\item Find a $b$-balanced cut (both sides $\ge bn$ vertices for some $b=\Omega(1)$) with expansion $< 1/4$. Let $T$ be smaller side. (This differs from KRV, which always finds \emph{sparsest} $b$-balanced cut.)
\item Add arbitrary vertices until $T$ has size $n/2$
\item Add arbitrary matching between $T,\overline T$ into $G$
\item Repeat while there exists $b$-balanced cut with expansion $<1/4$
\item Do extra stuff to ensure no unbalanced sparse cuts too (irrelevant for our case)
\end{enumerate}

\subsection{Proof strategy}
Define $p_{u,v}(t)$ as follows. We have matchings $M_1,\ldots,M_t$ in $G$ so far. Consider a random walk starting at $u$. For steps $i$ from $1$ to $t$ in that order, with probability $1/2$, walk along the edge in $M_i$ incident to the walk's current location. Let $p_{u,v}(t)$ be the probability that the walk (starting from $u$) ends at $v$. We define the potential function
\[ \Phi(t) := \sum_u \ent(p_{u,\cdot}(t)) = \sum_u \left( -\sum_v p_{u,v}(t)\log p_{u,v}(t)\right) .\]
Observe that
\begin{enumerate}
\item $\Phi(t) \ge 0$, since entropy is always nonnegative.
\item $\Phi(t) \le n\log n$, since each $\ent(p_{u,\cdot}(t))$ is at most $\log n$.
\item $\Phi(t)\ge\Phi(t-1)$. We show the stronger statement $\ent(p_{u,\cdot}(t)) \ge \ent(p_{u,\cdot}(t-1))$ for all $u\in V$. This is intuitively because (1) the walk at step $t$ averages, for each edge in matching $M_i$, the values of $p_{u,v}(t-1)$ at the two matched vertices, and (2) entropy function is concave.
\end{enumerate}
We'll show that $\Phi(t)$ increases by $\Omega(n)$ whenever we find a $b$-balanced cut with expansion $<1/4$.

From now on, let's re-define $b$ so that $|T|=bn$, i.e., the smaller side of the cut has exactly $bn$ vertices. Define $q_u:=\sum_{v\in\overline T}p_{u,v}(t)$, the probability that a walk starting from $u$ ends up in $\overline T$.

\begin{claim}\label{clm:1}
$\sum_{u\in T}q_u(t)<bn/8$. That is, on average, a random walk starting at a vertex in $T$ should have a small chance $(1/8)$ of ``escaping'' to $\overline T$.
\end{claim}
\begin{proof}
There are less than $bn/4$ edges crossing $T$. We want to show that each edge is ``responsible'' for a total of $1/2$ probability inside $\sum_{u\in T}q_u(t)$. This means that $\sum_{u\in T}q_u(t)\le(1/2)\cdot |E(T,\overline T)|<bn/8$.

Imagine running each random walk starting from each $u\in T$ in parallel. We can couple the walks so that at any given time, there is a single random walk currently at each vertex in $V$. This means that for all $v\in V$, $\sum_u p_{u,v}(t)=1$. Now, for each edge inside $E(T,\overline T)$ (say, $(u,v)$ where $u\in T,v\in\overline T$), consider the matching $M_i$ that it belongs to. The edge is responsible for mixing $1/2$ of the total of $1$ mass at $u$ into $v$ at step $i$ of the random walks. Note that the inequality $\sum_{u\in T}q_u(t)\le(1/2)\cdot |E(T,\overline T)|$ is because there may be some chance that a random walk goes out of $T$ (and therefore charged to some edge in $E(T,\overline T)$ but then returns back to $T$.
\end{proof}

Fix a vertex $u$ such that $q_u(t)\le1/4$, i.e., a random walk starting from $u$ escapes with probability at most twice the average of $1/8$. By \Cref{clm:1} and an averaging argument, there are $\ge bn/2$ vertices $u\in T$ satisfying this. Our next goal is to show that for each such $u$, the vertices $v\in T$ with $p_{u,v}(t) \ge 2p_{u,\pi(v)}(t)$, where $\pi(v)$ is the matched edge of $v$ in $M_t$ (and therefore $\pi(v)\in\overline T$ by construction), make up at least a constant fraction of the total probability by $p_{u,v}(t)$. It is easy to see (algebra) that if we mix two probabilities $p,q$ with $p\ge2q$, then the entropy increase
\[ -\left( 2 \cdot \left(\frac{p+q}2\right)\log\left(\frac{p+q}2\right)\right) - \left( -p\log p-q\log q \right) \]
is $\Omega(p)$. So it suffices to find a constant fraction of vertices $v\in T$ satisfying $p_{u,v}(t)\ge2p_{u,\pi(v)}(t)$.

Recall that every vertex in $T$ is matched to a vertex in $\overline T$. Call a vertex $v\in T$ ``good'' if $p_{u,v}(t) \ge 2p_{u,\pi(v)}(t)$, and ``bad'' otherwise. We have $q_u(t) =\sum_{v\in\overline T}p_{u,v}(t) \le 1/4$, so the total probability $p_{u,v}(t)$ of all bad $v\in T$ is at most $1/2$. Since this vertex $u$ has probability $\ge3/4$ inside $T$, the probability that it's inside $T$ and good is $\ge 3/4 - 1/2 = 1/4$. In other words, $\sum_{v\text{ good}}p_{u,v}(t)\ge1/4$. So we're done.

\section{Modifications}

Let's make the following two changes to the algorithm:
\begin{enumerate}
\item We find approximate $b$-balanced sparse cuts. In particular, if the graph has expansion $< \frac1{\textup{polylog}(n)}$, then we can find a $b$-balanced cut of expansion $<1/4$.
\item We don't augment $T$ into a set of size $n/2$. So we find a matching in $(T,\overline T)$ that saturates $T$ but doesn't necessarily saturate $\overline T$.
\end{enumerate}

I think both of these changes are fine. Note that for (1), the potential function proof of CMG makes no use of the conductance of $G$. That is, no matter the current conductance of $G$, as long as we can find a $b$-balanced cut of sparsity $<1/4$, then $\Phi(t)$ increases by $\Omega(n)$. For (2), some vertices in each $M_i$ are now no longer matched, but we can re-define the random walks so that if a walk is currently at an unmatched vertex at step $i$, then stay there with probability $1$. This allows us to again couple the random walks starting from each $u\in V$, so that every edge in $E(T,\overline T)$ is responsible for $1/2$ probability out.


\end{document}
